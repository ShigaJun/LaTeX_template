\documentclass[dvipdfmx]{jsarticle} % 文書クラスの設定
\usepackage[T1]{fontenc} % フォントの設定
\usepackage{lmodern} % フォントの設定
% この記事〈https://qiita.com/zr_tex8r/items/297154ca924749e62471〉に,フォント設定の詳細な説明がまとまっていました.
\usepackage{multicol} % 多段組
\usepackage{amsthm,amsmath,amssymb,amsfonts} % ams系
\usepackage{latexsym} % 数式で使える記号を増やす
\usepackage{mathrsfs} % 花文字など
\usepackage{mathtools} % 数学系のいろいろ
\usepackage[dvipdfmx]{graphicx,xcolor} % グラフィックと色
\usepackage{float,wrapfig} % 図表の配置
\usepackage{booktabs,multirow} % 表
\usepackage{appendix} % 付録
\usepackage{plistings} % ソースコードで日本語を利用(https://github.com/h-kitagawa/plistings)
\usepackage[dvipdfmx]{hyperref} % ハイパーリンク

\makeatletter % \makeatletter ... \makeatother で囲むことで,内部カウンタの利用を可能にする
\hypersetup{% hyperrefオプションリスト
  setpagesize=false,% PDF のページサイズを自動的に設定しない(jsarticle では不要)
  bookmarksdepth=\the\c@tocdepth,% PDF のしおりの階層を目次の深さに合わせる
}
\makeatother

\theoremstyle{definition} % 定理環境のスタイル設定
\newtheorem{theorem}{定理}[section]
\newtheorem{proposition}[theorem]{命題}
\newtheorem{lemma}[theorem]{補題}
\newtheorem{corollary}[theorem]{系}
\newtheorem{definition}[theorem]{定義}
\newtheorem{remark}[theorem]{注}
\newcommand{\x}{$\mathbb{X}$} % マクロの作成(新しいコマンドの定義)

% タイトルの設定
\title{タイトル}
\author{著者}
% \author{太郎\thanks{○大学} \and 次郎\thanks{●大学院} \and 三郎\thanks{株式会社△}} % 著者が複数,注意書きが必要の場合
\date{\today}

\begin{document} % 本文の開始

\maketitle % タイトルを出力

\begin{abstract} % 概要
  概要は,論文全体の内容を簡潔に伝える部分であり,読者が研究の主旨を素早く理解できるようにするための重要な部分である.概要を書く際には,まず研究の背景を簡潔に述べ,なぜこの研究が重要であるのかを伝える.次に,研究の目的を明確にし,この研究が何を解決しようとしているのかを示す.その後,使用した主要な方法を簡単に紹介し,研究のアプローチを読者に理解させる.重要な結果を要約し,研究の成果を簡潔に伝えることが求められる.最後に,研究の結論とその意義を簡潔にまとめ,研究がどのように新たな知見を提供したのか,または今後の研究にどのように貢献するのかを説明する.概要では過度に詳細な説明は避け,論文全体のエッセンスを簡潔に示すことが大切である.
\end{abstract}

\hrulefill % 罫線

\setcounter{tocdepth}{3} % 小小節まで目次に表示
\tableofcontents % 目次を出力

\newpage % 改ページ

% \begin{multicols}{2} % 2段組の開始
和文雑誌
%%%%%%%%%% 実際に利用する際には,ここから削除してください. %%%%%%%%%%

\part{部}
% \chapter{章} % bookとreportクラスのみ有効
\section{節}
\subsection{小節}
\subsubsection{小小節}
\paragraph{段落}
\subparagraph{小段落}

\part{\LaTeX の基本コマンド}
以下に,基本的な\LaTeX コマンドを紹介する.

\section{改行}
段落の改行→\par
↓「空白行」でも改行可能↓

段落内の強制改行→\\
改行後の字下げなし

\section{太字/斜体}
\textbf{太字 Bold}\par
\textit{斜体 Italic}

\section{箇条書き}
\paragraph{箇条書き}
\begin{itemize}
  \item 第1レベル
  \begin{itemize}
    \item 第2レベル
    \begin{itemize}
      \item 第3レベル
      \begin{itemize}
        \item 第4レベル
      \end{itemize}
    \end{itemize}
  \end{itemize}
\end{itemize}

\paragraph{番号付き箇条書き}
\begin{enumerate}
  \item 第1レベル
  \begin{enumerate}
    \item 第2レベル
    \begin{enumerate}
      \item 第3レベル
      \begin{enumerate}
        \item 第4レベル
      \end{enumerate}
    \end{enumerate}
  \end{enumerate}
\end{enumerate}

\section{数式}
\paragraph{インライン数式}
文章の中に数式$\displaystyle lim_{n \to \infty} \frac{1}{n} \sum^{n}_{k=1} f \left( \frac{k}{n} \right) = \int^{1}_{0} f(x)dx$を埋め込むことができる.

\paragraph{別行立て数式}
\begin{align}
  (a+b)^2 &= (a+b)(a+b)    \\
          &= a(a+b)+b(a+b) \\
          &= a^2+ab+ba+b^2 \\
          &= a^2+2ab+b^2 \label{eq:sample_formula}
\end{align}

\section{定理環境}
\begin{theorem}
任意の実数\( a, b, c \)に対して、\( a^2 + b^2 \geq c^2 \)である。
\end{theorem}

\begin{lemma}
実数において、\( a + b = c \)ならば、\( a \leq c \)である。
\end{lemma}

\begin{proof}
  自明
\end{proof}

\begin{definition}
\( a \)が定義域に属するならば、\( a \in A \)と表す。
\end{definition}

\begin{remark}
この証明は簡単な計算を使用しています。
\end{remark}

\section{参照}
\paragraph{基本的な参照}
式\ref{eq:sample_formula},図\ref{fig:sample_figure},表\ref{tab:sample_table}など.
\paragraph{参考文献の参照}
大学生は課題が終わると同時に,「次はもっと早くやる」と誓うが実行しない哲学者であることが知られている\cite{和文雑誌}.

\section{図・表}
\begin{figure}[H]
  \centering
  \includegraphics[width=50mm]{figures/sample.png}
  \caption{図のキャプション}
  \label{fig:sample_figure}
\end{figure}

\begin{table}[H]
  \caption{表のキャプション}
  \label{tab:sample_table}
  \centering
  \begin{tabular}{lcr}
    \hline
    データの型  & 宣言  &  ビット幅  \\
    \hline \hline
    文字型  & char  & 8 \\
    整数型  & int   & 32 \\
    倍精度実数型  & double  & 64 \\
    倍々精度実数型  &  long double  &  96 \\
    \hline
  \end{tabular}
\end{table}

\section{URL}
私の\x アカウント〈\url{https://x.com/wata_haru_4869}〉をしれっと宣伝する.
% 新しいコマンドの定義については,22行目の「マクロの作成」を参照
\href{https://watanabeharuto.netlify.app/}{マイサイト}\footnote{url}

%%%%%%%%%% 実際に利用する際には,ここまで削除してください. %%%%%%%%%%

\part{レポート・論文フォーマット} % 「部」は,普段のレポートレベルの文章量では使用しなくて良いと思います.
\section{序論}
\subsection{背景}
背景では,研究分野の現状や動向を示し,その中で未解決の課題を明確にすることが求められる.まずは,研究分野全体の広がりや重要性について述べ,その後,具体的な問題や課題に焦点を絞っていくと効果的である.この際,適切な文献を引用することで信頼性を高め,読者に対して客観的な根拠を示すことが重要である.さらに,背景を論理的に構成することで,研究が必要とされる理由を自然に理解させることを意識する必要がある.背景は単なる情報提供ではなく,研究の必要性を論理的に示すことを目的として記述するものである.

\subsection{目的}
目的では,背景で提示した課題に対して本研究がどのようにアプローチし,何を達成しようとしているのかを簡潔かつ明確に述べる.この小節では,研究の具体的なゴールを示すことが重要である.さらに,読者が論文全体の方向性を理解できるよう,研究の新規性や期待される成果に言及することも効果的である.ここでは,詳細な手法や結果については言及せず,あくまで論文の意図を示すことに留めることが望ましい.目的を簡潔に示すことで,読者は本論を読む際に研究の意図をつかみやすくなる.

\subsection{先行研究}
先行研究では,これまでに行われた関連する研究を紹介し,その成果や限界について述べる.ここでは,単なる既存研究の羅列ではなく,それらの研究同士の関連性を整理し,分野全体の進展を俯瞰することが重要である.また,既存の手法やアプローチにおける未解決の課題を指摘することで,本研究の新規性を強調する.さらに,先行研究との差別化を図り,本研究の独自性を示すことによって,研究分野における本論文の位置づけを明確にできる.この小節では,論文の信頼性を高めるためにも,適切な文献の引用が不可欠である.読者が分野の現状を理解できるよう丁寧に整理しつつ,本研究の意義を自然に浮かび上がらせる構成を心がけるべきである.

\section{方法}
方法では,研究を再現可能にすることを目的として,使用した手法や手順を詳細に記述する.読者が同じ条件下で同様の結果を得られるよう,具体的かつ正確な情報を提示することが求められる.実験や調査を行った場合は,使用した機器や設定,条件を明確に示し,データの収集方法についても記述する.また,分析方法を述べる際には,具体的な手法名やその適用の仕方を記載することで,論文全体の信頼性を高める.この節では,過度な説明を避けつつ,必要な情報を過不足なく伝えることを意識することが重要である.

\section{結果}
結果では,得られたデータを客観的に提示し,読者がその内容を正確に理解できるようにする.この際,データは図表を用いてわかりやすく示すことが有効である.図表には説明文を付け,本文中でも図表の内容について言及することで,読者がデータの重要性を把握しやすくなる.また,結果を述べる際には,主観を含まず,観察された事実のみを簡潔に記述することが大切である.読者に対して結果を正確に伝え,次の考察でそれを解釈するための基盤を提供することを目的とする.

\section{考察}
考察では,結果に基づいてその解釈を行い,研究の意義を論理的に議論する.感想や個人的な意見は排除し,結果から読み取れる以上のことを述べないようにする.結果を背景や目的に照らし合わせて解釈し,その結果がどのような意義を持つのかを説明する.また,先行研究との比較を通じて,研究が新たな知見を提供したことを示し,研究の貢献を強調する.

\section{結論}
\subsection{まとめ}
まとめでは,研究の主な成果とその意義を簡潔に述べる.ここでは,研究目的がどのように達成されたのか,主要な結果がどのように得られたのかを簡潔に振り返る.研究の貢献を強調し,結果の解釈や重要性を再確認することで,論文全体の内容を読者に再認識させることが求められる.感情的な表現は避け,客観的かつ明確に記述することが重要である.過去の節で述べた内容を簡潔にまとめ,研究の主要なポイントを整理する.

\subsection{今後の展望}
今後の展望では,研究の限界や未解決の問題について言及し、それらに対する今後の研究の方向性を示す.この小節では,研究の成果をもとに,どのような追加の実験や調査が必要なのか,どの分野で更なる知見を得ることができるのかを提案する.また,研究の結果が将来的にどのように応用される可能性があるのかを考察することも重要である.今後の研究への期待や挑戦するべき課題を示すことで,読者に未来に向けたビジョンを提供する.

\section*{謝辞}
\addcontentsline{toc}{section}{謝辞} %番号のない節を目次に追加
本フォーマットを使用してくださり,ありがとうございます.

\begin{thebibliography}{99} % 参考文献
\addcontentsline{toc}{section}{参考文献}
  \bibitem{和文雑誌} 著者名:標題,雑誌名,Vol.xx,No.xx,pp.xx-xx(西暦年).
  \bibitem{英文雑誌} Author: Title, \textit{Journal Name}, Vol.xx, No.xx, pp.xx-xx (Year). % Authorは,著者のFamily nameのあとに","(コンマ)をつけて, Middle Name, Given nameのイニシャルのあとに"."(ピリオド)をつけてください.
  \bibitem{国際会議プロシーディング} Author: Title, Proc.\textit{Proceedings Name}, Editor's name, pp.xx-xx, Publisher (Year).
  \bibitem{単行本} 著者:タイトル,pp.xx-xx,版元(西暦年).
  \bibitem{訳本} Original Author: Original Title (Original Year). 訳者名(訳):訳本タイトル,pp.xx-xx,版元(西暦年).
  \bibitem{電子雑誌} 標題,雑誌名,Vol.xx,No.xx,pp.xx-xx(オンライン),DOI: 10.xxxx/xxxxxx(西暦年). % DOIについてはここから〈https://www.doi.org/the-identifier/resources/handbook/〉
  \bibitem{Webサイト,Webページ} 著者名:Webページの題名,Webサイトの名称(オンライン),入手先〈\url{https://xxxxxxxxxx}〉(参照20xx-xx-xx).
\end{thebibliography}

\appendix % 以降の節が付録になる

\section{ソースコード}

% ソースコードの表示設定
\lstset{
  frame=single, % 枠で囲む
  numbers=left, % 左に行番号を表示
  breaklines = true % 長い行の改行
}

% ソースコードの挿入
\begin{lstlisting}[language=Python,caption=hello\_world.py,label=code:hello_world]
# "Hello World!" と出力するプログラム
message = "Hello World!"
print(message)
\end{lstlisting}

% \end{multicols} % 2段組の終了

\end{document}